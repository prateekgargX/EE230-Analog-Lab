<<<<<<< HEAD
% XCircuit output "RC_Integrator.eps.tex" for LaTeX input from RC_Integrator.eps.ps
\def\putbox#1#2#3{\makebox[0in][l]{\makebox[#1][l]{}\raisebox{\baselineskip}[0in][0in]{\raisebox{#2}[0in][0in]{#3}}}}
\def\rightbox#1{\makebox[0in][r]{#1}}
\def\centbox#1{\makebox[0in]{#1}}
\def\topbox#1{\raisebox{-\baselineskip}[0in][0in]{#1}}
\def\midbox#1{\raisebox{-0.5\baselineskip}[0in][0in]{#1}}
\begin{flushleft}
\includegraphics[width=\textwidth]{RC_Integrator.eps}\\
% translate x=624 y=118 scale 0.38
\putbox{1.31in}{0.84in}{R=10K$\Omega$}%
\putbox{1.81in}{0.25in}{C=0.1$\mu$F}%
\putbox{0.06in}{0.42in}{$V_{in}$}%
\putbox{0.47in}{0.50in}{5v}%
\putbox{0.89in}{0.25in}{0v}%
\end{flushleft}
=======
\begin{center}
    \begin{circuitikz}[american voltages]
        \draw
        (0,0) to [short, o-o] (6,0)
        to [open, v>=$V_{out}$] (6,4) 
        (0,0) to [open, v^>=$V_{in}$,o-o] (0,4)  
        to [R, l= $10K \Omega $,o-o] (6,4)
        (5,4) to [C, l_=$0.1\mu F$,*-*] (5,0); 
    \end{circuitikz}

    RC Integrator Circuit
\end{center}

>>>>>>> circuitikz
